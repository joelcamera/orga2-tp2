\subsubsection{Cropflip - Versión 1/2/3 - ASM}

Las implementaciones de ASM trabajan sobre la imagen como un vector en vez de una matriz.
Para esto se manejan dos punteros. El de destino apuntando al principio de la ultima fila. 
Y en el de origen se saltean de la posición de memoria de inicio el offsetY * anchoFila * bytesPorPixel.

El ciclo de repite tamY veces y realiza lo siguiente:
\begin{enumerate}

\item Avanza el puntero de origen offsetX * bytesPorPixel.
\item Se copian del origen al destino la cantidad tamX de pixels.
\item Se saltean los pixels no copiados de la fila. 
	{\\ ((anchoDeFila - tamX - offsetX) * bytesPorPixel)}

\end{enumerate}

Las 3 versiones de ASM difieren en el punto 2 que es la forma de copiar los pixels. 
La versión 1 utiliza rep movsq (que repite la copia de quadwords)
La versión 2 realiza un ciclo copiando manualmente de [SRC] a un registro y luego a [DST]
La versión 3 realiza lo mismo que la 2 pero utilizando un registro XMM y por ende copiando de a 16 bytes.