\subsubsection{Sepia - Versión 1 - ASM}

Las implementaciones en ASM, en donde utilizamos SIMD, son muy parecidas con respecto a como se copian los valores del origen y como se depositan en destino, solo cambian en la forma de realizar los cálculos.

En nuestra primera implementación de \textit{Sepia} en ASM (\textbf{sepia_asm.asm}) pasamos cuatro pixels de forma empaquetada al registro XMM0 y, realizando la operación \textbf{movdqa}, pasamos los valores a XMM2, XMM3 y XMM4.

\begin{table}[h]
	\centering
	\begin{tabular}{| c | c | c | c | c | c | c | c | c | c | c | c | c | c | c | c |}
		\hline
		a3 & r3 & g3 & b3 & a2 & r2 & g2 & b2 & a1 & r1 & g1 & b1 & a0 & r0 & g0 & b0  \\ \hline
		\multicolumn{16}{c}{XMM0 $=$ XMM2 $=$ XMM3 $=$ XMM4} \\
	\end{tabular}
\end{table}

Luego, se realiza un shuffle de a bytes con la operación \textbf{pshufb} aplicando mascaras a los registros XMM2, XMM3 y XMM4 dejando los registros de la siguiente forma:

\begin{tabbing}
	\textit{maskB}: \= DB \= 0x00,0xFF,0xFF,0xFF, \= 0x04,0xFF,0xFF,0xFF, \= 0x08,0xFF,0xFF,0xFF, \= 0x0C,0xFF,0xFF,0xFF \\
	\textit{maskG}: \= DB \= 0x01,0xFF,0xFF,0xFF, \= 0x05,0xFF,0xFF,0xFF, \= 0x09,0xFF,0xFF,0xFF, \= 0x0D,0xFF,0xFF,0xFF \\
	\textit{maskR}: \= DB \= 0x02,0xFF,0xFF,0xFF, \= 0x06,0xFF,0xFF,0xFF, \= 0x0A,0xFF,0xFF,0xFF, \= 0x0E,0xFF,0xFF,0xFF \\
\end{tabbing}

\begin{table}[h]
	\centering
	\begin{tabular}{| c | c | c | c | c | c | c | c | c | c | c | c | c | c | c | c |}
		\hline
		0 & 0 & 0 & b3 & 0 & 0 & 0 & b2 & 0 & 0 & 0 & b1 & 0 & 0 & 0 & 
		b0  \\ \hline
		\multicolumn{16}{c}{XMM2} \\
		\hline
		0 & 0 & 0 & g3 & 0 & 0 & 0 & g2 & 0 & 0 & 0 & g1 & 0 & 0 & 0 & g0  \\ \hline
		\multicolumn{16}{c}{XMM3} \\
		\hline
		0 & 0 & 0 & r3 & 0 & 0 & 0 & r2 & 0 & 0 & 0 & r1 & 0 & 0 & 0 & r0  \\ \hline
		\multicolumn{16}{c}{XMM4} \\
	\end{tabular}
\end{table}

Sumando estos tres registros entre si con la operación \textbf{paddd} obtengo en el registro XMM2 la \textbf{suma} de los valores de cada píxel.

\begin{table}[!h]
	\centering
	\begin{tabular}{| c | c | c | c |}
		\hline
		b+g+r 3 & b+g+r 2 & b+g+r 1 & b+g+r 0 \\ \hline
		\multicolumn{4}{c}{XMM2}
	\end{tabular}
\end{table}


Con la operación \textbf{cvtdq2ps} hacemos la conversión de la suma a punto flotante empaquetado. Luego realizamos una división por diez a cada uno de los valores del registro XMM2 con la operación \textbf{divps}, esto es lo mismo que se los multiplique por $0.1$.

\begin{table}[!h]
	\centering
	\begin{tabular}{| c | c | c | c |}
		\hline
		suma$*0.1$ 3 & suma$*0.1$ 2 & suma$*0.1$ 1 & suma$*0.1$ 0 \\ \hline
		\multicolumn{4}{c}{XMM2}
	\end{tabular}
\end{table}

Moviendo el valor a los registros XMM3 y XMM4 y realizando sumas entre sí con la operación \textbf{addps} que suma packed obtenemos en el registro XMM2 la suma multiplicada por $0.2$, en el registro XMM3 la suma multiplicada por $0.3$ y en el registro XMM4 la suma multiplicada por $0.5$ que representarían a los componentes azul, verde y rojo respectivamente.

\begin{table}[!h]
	\centering
	\begin{tabular}{| c | c | c | c |}
		\hline
		suma$*0.2$ 3 & suma$*0.2$ 2 & suma$*0.2$ 1 & suma$*0.2$ 0  \\ \hline
		\multicolumn{4}{c}{XMM2} \\
		\hline
		suma$*0.3$ 3 & suma$*0.3$ 2 & suma$*0.3$ 1 & suma$*0.3$ 0  \\ \hline
		\multicolumn{4}{c}{XMM3} \\
		\hline
		suma$*0.5$ 3 & suma$*0.5$ 2 & suma$*0.5$ 1 & suma$*0.5$ 0  \\ \hline
		\multicolumn{4}{c}{XMM4}
	\end{tabular}
\end{table}


Luego, se vuelve a convertir estos registros con la operación \textbf{cvtps2dq} a double words y utilizando las operaciones \textbf{packusdw} y luego \textbf{packuswb} se los vuelve a convertir a byte con saturación sin signo quedando en el registro xmm2 los 16 bytes de los valores.

\begin{table}[!h]
	\centering
	\begin{tabular}{| c | c | c | c | c | c | c | c | c | c | c | c | c | c | c | c |}
		\hline
		r'3 & r'2 & r'1 & r'0 & r'3 & r'2 & r'1 & r'0 & g'3 & g'2 & g'1 & g'0 & b'3 & b'2 & b'1 & b'0  \\ \hline
		\multicolumn{16}{c}{XMM2} \\
	\end{tabular}
\end{table}

Donde r'x $=$ \textbf{suma}$* 0.5$, g'x $=$ \textbf{suma}$* 0.3$ y b'x $=$ \textbf{suma}$* 0.2$. Las 'x' representan el pixel al que pertenecen.

Realizando un shuffle sobre este registro reordeno los componentes y dejando las coordenadas $\alpha$ en el registro XMM0 con una simple operación de \textbf{por} entre ambos registros obtengo en XMM0 los valores de cada píxel en su lugar para depositarlo en la memoria.

\begin{tabbing}
	\textit{reord}: \= DB \= 0x0,0x4,0x8,0xFF, \= 0x1,0x5,0x9,0xFF, \= 0x2,0x6,0xA,0xFF, \= 0x3,0x7,0xB,0xFF \\
	\textit{maskA}: \= DB \= 0x00,0x00,0x00,0xFF, \= 0x00,0x00,0x00,0xFF, \= 0x00,0x00,0x00,0xFF, \= 0x00,0x00,0x00,0xFF \\
\end{tabbing}

\begin{table}[!h]
	\centering
	\begin{tabular}{| c | c | c | c | c | c | c | c | c | c | c | c | c | c | c | c |}
		\hline
		a3 & r'3 & g'3 & b'3 & a2 & r'2 & g'2 & b'2 & a1 & r'1 & g'1 & b'1 & a0 & r'0 & g'0 & b'0  \\ \hline
		\multicolumn{16}{c}{XMM0} \\
	\end{tabular}
\end{table}